Efficient metabolism in bioengineered tissues requires a robust vascular system to provide healthy microenvironments to the cells and stroma. Such networks form spontaneously during embryogenesis from randomly distributed endothelial cells. There is a need to bioengineer endothelial cells so that network formation and operation is optimal for synthetic tissues. This work introduces a computational model that simulates \textit{de novo} vascular development and assesses the effectiveness of the network in delivering nutrients and extracting waste from tissue. A genetic algorithm was employed to identify parameter values of the vaculogenesis model that lead to the most efficient and robust vascular structures. These parameter values control the behavior of cell-level mechanisms such as chemotaxis and adhesion. These studies demonstrate that genetic algorithms are effective at identifying model parameters that lead to near-optimal networks. This work suggests that computational modeling and optimization approaches may improve the effectiveness of engineered tissues by suggesting target cellular mechanisms for modification. 
