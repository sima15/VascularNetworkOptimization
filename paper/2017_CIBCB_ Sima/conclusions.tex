This work has shown that genetic algorithms are sufficiently powerful to optimize complex biological systems, even when there is a non-linear relationship between the parameters and the fitness function. This method can be applied to models of different biological systems that consider alternative objectives.

Many challenges exist before this optimization approach can be applied to actual bioengineered tissues. Two principle challenges are that (a) the speed of the simulations needs to be increased to allow expansion of the parameter space and to increase the fidelity of the model, and (b) methods to engineer the molecules that influence vascular tissue organization to meet the predicted optimal parameter values must be expanded and improved. However, although not all the molecules considered in this work can yet be easily tuned to meet optimal parameter values, existing real-world approaches are available for at least some of them. For example, the secretion rate of the chemoattractants can potentially be altered by changing the intracellular stability of genetically engineered chemoattractants guided by the N-end rule \cite{Varshavsky2011Nend}.  Similarly, there is the possibility of increasing the decay rate of both the short-range and long-range chemoattractants by adding selective targets of specific proteases that could be introduced into the culture system. Exploring these modifications may move this in silico system to an in vitro system with practical applications.

