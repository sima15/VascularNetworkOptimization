
One of the main challenges in the computational modeling of biological systems is the determination of the models' parameters. This problem is particularly acute with multiscale models that are gaining in popularity due to their realism. In this work, we identified the parameter values of a cellular regulatory mechanism using spatiotemporal multicellular data. While the problem of finding parameter values that describe the VEGF autoregulatory mechanism of RPE cells is simple, this problem domain serves and a proof-of-concept for the overall method. Most importantly, the method demonstrates that it is possible to utilize data at one scale to determine parameter values at a different scale.

In the work presented here, thousands of simulations were performed as the search method explored the parameter space. For each potential solution, multiple simulations were needed over each experimental case (different patch sizes) and because of the need for repeats due to model stochasticity. For the simple 2D RPE model, each simulation took less than a minute and so the process could be completed quickly. In general, the method can rapidly become infeasible as the number of unknown parameters grows, domains become larger and more complex, and the number of specific experimental cases grows.

A recently developed hybrid simulation system called Biocellion \cite{Kang2014Biocellion} could be used instead of the iDynamics to speedup the rate of simulations. Biocellion can utilize thousands of processing nodes and rapidly simulate complex models of billions of cells. The search method utilized in this paper was chosen for its simplicity and insights on the local error surfaces provided by the sweeping process. However, the fitting method is independent of the method employed to search the parameter space. Alternative methods of combinatorial optimization may improve performance.

One problem all fitting methods must deal with is under or over fitting the model. There is a possibility that the fitting problem may be under-constrained for lack of data. This problem will be explored in the RPE domain by expanding the data set to include additional studies with VEGF agonists and alternative pattern arrangements.
 
To extend the method to other domains, alternative error functions can be employed that measure discrepancies over a diversity of spatiotemporal features which quantify both the experimental observations and simulator outcomes. For instance, in \cite{Ruusuvuori2014Quantitative} image processing is applied to bright field time-lapse images of growing yeast colonies to extract trajectories of many visual features including volume, roughness, dominant frequency etc. The same features could be extracted from the morphologies of simulated colonies and used to fit parameters of the yeast model.

