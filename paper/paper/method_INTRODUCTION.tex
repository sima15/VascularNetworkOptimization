

An important aspect of computational systems biology is the investigation of dynamic biological processes that operate across multiple temporal and spatial scales by constructing and running multiscale models \cite{Kang2014Biocellion}, \cite{qanitabaker:Materi2007Computational}, \cite{qanitabaker:DiazOchoa2012Multiscale}, \cite{qanitabaker:Castiglione2014Modeling}, \cite{Vicini2010Multiscale}. These models incorporate a set of parameters that represent the physical and chemical properties of the biological system \cite{qanitabaker:Mahoney2012Multiobjective}. The parameters are used to define the components of the models that when simulated reproduce the behavior of the biological system. Often the correct values of these parameters are unknown or difficult to obtain \cite{qanitabaker:Walker2007Challenges}, \cite{qanitabaker:Transtrum2009Why}.

Recently, there has been an increase in the number of model-fitting methods proposed to estimate model parameters' values \cite{qanitabaker:RodriguezFernandez2006Novel}, \cite{qanitabaker:Chou2009Recent}, \cite{qanitabaker:Sun2012Parameter} from experimental data.  Without accurate estimations of parameters, predictions from simulation studies will most likely be erroneous and provide little scientific insight and guidance in disease treatment  \cite{qanitabaker:Banga2008Optimization}. This scenario can be ameliorated by fitting the model to experimental \textit{in vitro} / \textit{in vivo} data \cite{qanitabaker:BalsaCanto2008Hybrid} \cite{qanitabaker:Lillacci2010Parameter}. Finding the best-fit values for the unknown parameters enhances the possibility of performing accurate quantitative predictions.

 Vascular endothelial growth factor (VEGF) is a key promoter of angiogenesis and vascular development and is the target in numerous anti-angiogenic therapies \cite{Carmeliet2000Angiogenesis}. Angiogenesis is the growth of blood vessels from the preexisting vasculature, a process involved in the physiological functions of several diseases, such as cancer and age-related macular degeneration (AMD). Moreover, in spite of substantial basic science and translational research to develop anti-angiogenic therapies, many questions remain about the mechanisms of action of angiogenic drugs, how and why several diseases such as AMD become resistant to the treatment, or the patient conditions that can benefit most from these drugs \cite{Bergers2008Modes}. For these reasons, computational models of angiogenesis have been developed to simulate the process and provide a framework for generating and testing hypotheses of VEGF-driven processes \cite{qanitabaker:MacGabhann2006Computational, qanitabaker:Stefanini2008Compartment, qanitabaker:Gabhann2012Simulating}. Models have aided in the development of novel and effective anti-angiogenic therapeutics that target VEGF regulation and receptors \cite{qanitabaker:Mahoney2012Multiobjective}, \cite{qanitabaker:Finley2013Compartment,qanitabaker:Yen2011TwoCompartment}. Advancing these computational approaches combined with progress in \textit{in vitro} experimental studies will shed light on these issues by providing an effective framework for generating and testing hypotheses related to VEGF regulation and transport in the tissue \cite{qanitabaker:Stefanini2008Compartment}.

 An essential mechanism for understanding VEGF's role in disease development is its auto-regulation. The rate of VEGF secretion is controlled through an auto-inhibitory regulatory mechanism where the VEGF concentration of a cell's microenvironment down-regulates the secretion of VEGF. This control loop enables a community of cells to maintain a stable background concentration of VEGF \cite{Takahashi2005Vascular}. Disruption of the loop is implicated in multiple disease states.

 This paper presents a method for accurately characterizing this auto-regulation, not from microfluidic assays that interrogate individual or mixed cell populations but from spatially organized multicellular experimental data sampled over time. As will be explained later, spatiotemporal data provides unique insights because auto-regulation is inherently a mechanism that is manifested over space and time. The rest of the paper is structured as follows: First the experimental setup and computational model is described, along with the specific autoregulatory parameters that are known and those to be estimated. Second, the search method for finding the values for the parameters is described in detail. Next the method is evaluated by validating the identified parameter values. Finally, a summary of the method's effectiveness and suggestions for future work are given.

\section{Multicellular Experiment and Model}

Describe the study here.




