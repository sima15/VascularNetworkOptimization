

An important aspect of computational systems biology is the investigation of dynamic biological processes that operate across multiple temporal and spatial scales by constructing and running multiscale models \cite{Kang2014Biocellion}, \cite{qanitabaker:Materi2007Computational}, \cite{qanitabaker:DiazOchoa2012Multiscale}, \cite{qanitabaker:Castiglione2014Modeling}, \cite{Vicini2010Multiscale}. These models incorporate a set of parameters that represent the physical and chemical properties of the biological system \cite{qanitabaker:Mahoney2012Multiobjective}. The parameters are used to define the components of the models that when simulated reproduce the behavior of the biological system. Often the correct values of these parameters are unknown or difficult to obtain \cite{qanitabaker:Walker2007Challenges}, \cite{qanitabaker:Transtrum2009Why}.

Recently, there has been an increase in the number of model-fitting methods proposed to estimate model parameters' values \cite{qanitabaker:RodriguezFernandez2006Novel}, \cite{qanitabaker:Chou2009Recent}, \cite{qanitabaker:Sun2012Parameter} from experimental data.  Without accurate estimations of parameters, predictions from simulation studies will most likely be erroneous and provide little scientific insight and guidance in disease treatment  \cite{qanitabaker:Banga2008Optimization}. This scenario can be ameliorated by fitting the model to experimental \textit{in vitro} / \textit{in vivo} data \cite{qanitabaker:BalsaCanto2008Hybrid} \cite{qanitabaker:Lillacci2010Parameter}. Finding the best-fit values for the unknown parameters enhances the possibility of performing accurate quantitative predictions.

 Vascular endothelial growth factor (VEGF) is a key promoter of angiogenesis and vascular development and is the target in numerous anti-angiogenic therapies \cite{Carmeliet2000Angiogenesis}. Angiogenesis is the growth of blood vessels from the preexisting vasculature, a process involved in the physiological functions of several diseases, such as cancer and age-related macular degeneration (AMD). Moreover, in spite of substantial basic science and translational research to develop anti-angiogenic therapies, many questions remain about the mechanisms of action of angiogenic drugs, how and why several diseases such as AMD become resistant to the treatment, or the patient conditions that can benefit most from these drugs \cite{Bergers2008Modes}. For these reasons, computational models of angiogenesis have been developed to simulate the process and provide a framework for generating and testing hypotheses of VEGF-driven processes \cite{qanitabaker:MacGabhann2006Computational, qanitabaker:Stefanini2008Compartment, qanitabaker:Gabhann2012Simulating}. Models have aided in the development of novel and effective anti-angiogenic therapeutics that target VEGF regulation and receptors \cite{qanitabaker:Mahoney2012Multiobjective}, \cite{qanitabaker:Finley2013Compartment,qanitabaker:Yen2011TwoCompartment}. Advancing these computational approaches combined with progress in \textit{in vitro} experimental studies will shed light on these issues by providing an effective framework for generating and testing hypotheses related to VEGF regulation and transport in the tissue \cite{qanitabaker:Stefanini2008Compartment}.

 An essential mechanism for understanding VEGF's role in disease development is its auto-regulation. The rate of VEGF secretion is controlled through an auto-inhibitory regulatory mechanism where the VEGF concentration of a cell's microenvironment down-regulates the secretion of VEGF. This control loop enables a community of cells to maintain a stable background concentration of VEGF \cite{Takahashi2005Vascular}. Disruption of the loop is implicated in multiple disease states.

 This paper presents a method for accurately characterizing this auto-regulation, not from microfluidic assays that interrogate individual or mixed cell populations but from spatially organized multicellular experimental data sampled over time. As will be explained later, spatiotemporal data provides unique insights because auto-regulation is inherently a mechanism that is manifested over space and time. The rest of the paper is structured as follows: First the experimental setup and computational model is described, along with the specific autoregulatory parameters that are known and those to be estimated. Second, the search method for finding the values for the parameters is described in detail. Next the method is evaluated by validating the identified parameter values. Finally, a summary of the method's effectiveness and suggestions for future work are given.

\section{Multicellular Experiment and Model}

The experiment from which the unknown parameter values are derived employs bioengineered micropatterning techniques. The micropatterns form a regular arrangement of circular 2D ‘patches’ populated with cells surrounded by an exposed substrate. The exposed regions emulate necrotic areas of the retinal tissue that result from repeated exposure to reactive oxidative species, triggering neovascularization and exudative AMD \cite{Chopdar2003Age}. Recreating these regular spatially organized cellular configurations is essential to understanding the impact of local cell-cell and cell-environment interactions on VEGF autoregulation. 

In the experimental study, described in \cite{qanitabaker:Vargis2014Effect}, the bioengineered circular micro patterns were employed to control the extent of cell-cell interactions, which occur within the patch, and cell-environment interactions, which occur at the perimeter. Several patch sizes were used in this study ( \SI{100}{\micro\metre}, \SI{200}{\micro\metre}, \SI{300}{\micro\metre}, and \SI{400}{\micro\metre}) to sample the proportion of cell-cell and cell-environment interactions in each experiment. Such sampling constrains the possible parameter values. Each patch was seeded with retinal pigment epithelial (RPE) cells and grown in a cellular culture. As the cells grew, the VEGF per cell was measured at regular intervals: 4, 24, 30 48, 54, 72 hours. To measure the VEGF per cell, enzyme-linked immunosorbent assay (ELISA) was used to determine the total VEGF contained within the cell culture, and the number of cells per patch was determined by image analysis proceeded by staining. Figure~\ref{in_vitro_experiment}(a) (taken from \cite{qanitabaker:Vargis2014Effect}) illustrates the stained patches at 72 hours. Experiments were repeated ten times and averaged. The final spatiotemporal data produced is illustrated in Figure~\ref{in_vitro_experiment}(b) and forms the target prediction for the computational model simulation.

The bioengineered experiments were simulated using a hybrid agent-based approach, which is an extension of \textsl{iDynoMiCs} framework developed by the Kreft group at University of Birmingham \cite{qanitabaker:Lardon2011IDynoMiCS}. This model was selected because of its extensibility and easy of use. All inputs to the model such as parameter values and initial condition are easily specified using an XML document called the protocol file. Hybrid models integrate discrete components to represent the cells and continuous equations to represent biochemical reactions and diffusion. Each cell is a spherical particle that grows by consuming nutrient and accumulating biomass volume; when the volume exceeds twice the initial volume, cell division is simulated by splitting the particle into two. Particles can secrete and uptake soluble biochemicals (such as VEGF) which diffuse through the domain; regulatory reactions that model interactions among intracellular and inter-cellular proteins become PDEs. The simulation interlaces cellular growth and movement (implemented by relaxing forces between particles) with biochemical redistribution (implemented by solving the PDEs). Random noise disrupts cellular movement and the division volume to represent the inherent stochasticity of the biological processes.

The setup of the simulations replicate the experimental conditions and units of the \textit{in vitro} experiments. The 2D domain size of each simulation is \SI{2400}{\micro\metre} by \SI{2400}{\micro\metre}, initial cell size is set to $80 \mu m^2$ and the doubling time due to growth is set at 36 hours. Each simulation begins with multiple RPE cells distributed randomly at the same density and with the patch pattern. The simulation replicates the first 72 hours of the \textit{in vitro} experiments. Illustrations of the simulated experiments are shown in Figure~\ref{simulation_examples}.

This framework is inherently multiscale in that the parameters that control the low-level mechanisms at the cellular level, e.g., growth, the VEGF secretion rate and autoregulation, determine the cell population and VEGF concentration over the complete multicellular domain. Figure~\ref{simulation_examples} illustrate the VEGF distributions in the domain. To compute the VEGF concentration per cell, the total VEGF is computed over the whole domain, while the number of cells is directly determined by the simulator. This approach intrinsically includes the quantitative spatiotemporal control effects as the cells grow, secret VEGF which diffuses over the domain. Moreover, the simulations provide insight into the spatial VEGF gradients within and between patches, unavailable in \textit{in vitro} studies.

%\cite{qanitabaker:Cao2007Spatiotemporal}

% \begin{figure}[!t]
%  \centering
%
% \begin{subfigure}{.5\textwidth}
%   \centering
%   \includegraphics[width=.8\linewidth]{./figures/in_vitro_crop.png}
%   \caption{(a) Patches of stained RPE cells at 72 hours for each patch size. \cite{qanitabaker:Vargis2014Effect}.}
%
% \end{subfigure}%
%
% \begin{subfigure}{.5\textwidth}
%   \centering
%   \includegraphics[width=.8\linewidth]{./figures/Results/In-Vitro.png}
%   \caption{(b) Time course of VEGF expression per cell measured at 4, 24, 30, 48, and 72 h ( data for each time from the \textit{in vitro} \cite{qanitabaker:Vargis2014Effect} ).}
% \end{subfigure}%
%\label{in_vitro_experiment}
%\end{figure}







\begin{figure}
 \begin{center}
  \begin{tabular}{cc}
   \includegraphics[width=0.40\textwidth]{./figures/in_vitro_crop.png} \\
   (a)\\
    \includegraphics[width=0.40\textwidth]{./figures/Results/In-Vitro.png} \\
   (b)   
   \end{tabular}
   \end{center}

\caption{(a) Patches of stained RPE cells at 72 hours for each patch size. \cite{qanitabaker:Vargis2014Effect}. (b) Time course of VEGF expression per cell measured at 4, 24, 30, 48, and 72 h ( data for each time from the \textit{in vitro} \cite{qanitabaker:Vargis2014Effect} ). }
  \vspace{+1mm}
\label{in_vitro_experiment}
\end{figure}


 The autoregulation of VEGF secretion is described in Equation~\ref{VEGF_autoregulation} as a function of biomass $M$ and local VEGF concentration $V$. The diffusion coefficient of VEGF, $D_{V}$, is set to $5.8 \times 10^{-11} m^{2} s^{-1}$ given in microfluidic experiments from \cite{qanitabaker:Shin2012Microfluidic}.

 \begin{equation}
 \frac{\partial V}{\partial t}=D_{V}\bigtriangledown^{2} V+ \mu _{V}  \frac{K}{ \beta V+K} M
 \label{VEGF_autoregulation}
 \end{equation}

Over time, the RPE cells grow based on a doubling time of 36 hours \cite{qanitabaker:Bryckaert2000Regulation}, which determines the growth rate parameter $\mu _{M}$. Since nutrient is unlimited and cell crowding is not an issue within the 72 hour time line, we applied first order kinetics for cell growth as shown in Equation \ref{Cell_Growth}.\\


\begin{equation}
\frac{\partial M}{\partial t}=   \mu _{M} M
\label{Cell_Growth}
\end{equation}

Table \ref{parameters} summarizes the description of the parameters used in the equations above and identifies the known parameters and those that need to be determined by the method introduced in this paper.

\begin{table}[ht]
\caption{ Known and unknown parameter descriptions} % title of Table
\centering
\begin{footnotesize}
\begin{tabular}{l l l}
\hline
Parameter   &  Value & Description\\ \hline \hline
%\\ [1ex]      % [1ex] adds vertical space
$D_{V}$     & $5.8 \times 10^{-11} m^{2} s^{-1}$ & Diffusion coefficient \\
$\mu _{M}$  &  1.0194   $ hour^{-1}$                 & Max. growth rate for RPE cells \\
[1ex]      % [1ex] adds vertical space
\hline
%\\ [1ex]      % [1ex] adds vertical space
$K$       &  \textsl{Unknown}                  & Auto regulation rate \\
$\mu _{V}$ & \textsl{Unknown}                   & Secretion rate \\
$\beta $    &  \textsl{Unknown}                  & Binding affinity \\
[1ex]      % [1ex] adds vertical space
 

 
 
\hline
\end{tabular}
\end{footnotesize}
\label{parameters}
\end{table}

\begin{figure}
 \begin{center}
  \begin{tabular}{cc}
   \includegraphics[width=0.20\textwidth]{./figures/400_one.png} &   \includegraphics[width=0.20\textwidth]{./figures/Patch400.png} \\
   (a) & (b) \\
   \includegraphics[width=0.20\textwidth]{./figures/O400VEGF solute 072.png} &  \includegraphics[width=0.20\textwidth]{./figures/C400VEGF solute 072.png} \\
   (c) & (d)
   \end{tabular}
   \end{center}

\caption{(a) Closeup of initial condition of a \SI{400}{\micro\metre} patch (b) Experiment with \SI{400}{\micro\metre}, 3 patches in each side. (c) Closeup of the VEGF distribution after 72 hours, (d) VEGF distribution over whole domain after 72 hours.  }
  \vspace{+1mm}
\label{simulation_examples}
\end{figure}





