

%\documentclass[journal,10pt,draftclsnofoot,onecolumn]{IEEEtran} %draftclsnofoot
%\documentclass[confl, draftclsnofoot]{IEEEtran}
%\documentclass[conference,onecolumn,draft]{IEEEtran}
\documentclass[conference]{IEEEtran}
\usepackage{grffile}
\usepackage{color}
\usepackage{graphicx}
\usepackage{cite}
\usepackage{multicol}
\usepackage{subcaption}
\usepackage{algorithm}
\usepackage{algpseudocode}
\usepackage{pifont}
\usepackage{subcaption}
\usepackage{siunitx}

%\usepackage{setspace}
%\doublespacing
%\onecolumn
\begin{document}
%
% paper title
% can use linebreaks \\ within to get better formatting as desired
\title{Bridging the Multiscale gap: Identifying Cellular Parameters from Multicellular Data\\}
%
%
% author names and IEEE memberships
% note positions of commas and nonbreaking spaces ( ~ ) LaTeX will not break
% a structure at a ~ so this keeps an author's name from being broken across
% two lines.
% use \thanks{} to gain access to the first footnote area
% a separate \thanks must be used for each paragraph as LaTeX2e's \thanks
% was not built to handle multiple paragraphs
%




\author{Qanita~Bani Baker, Gregory~J.~Podgorski, Christopher D. Johnson, Elizabeth Vargis, Nicholas~S.~Flann % <-this % stops a space
\IEEEcompsocitemizethanks{Qanita~Bani Baker, Gregory J. Podgorski, Elizabeth Vargis, and ~Nicholas~S.~Flann  are with Utah State University, Logan, Utah,  USA.
e-mail: (qanita@aggiemail.usu.edu, gregory.podgorski@usu.edu, christopher.d.johnson@aggiemail.usu.edu, elizabeth.vargis@usu.edu, and nick.flann@usu.edu)}}% <-this % stops a space


% note the % following the last \IEEEmembership and also \thanks -
% these prevent an unwanted space from occurring between the last author name
% and the end of the author line. i.e., if you had this:
%
% \author{....lastname \thanks{...} \thanks{...} }
%                     ^------------^------------^----Do not want these spaces!
%
% a space would be appended to the last name and could cause every name on that
% line to be shifted left slightly. This is one of those "LaTeX things". For
% instance, "\textbf{A} \textbf{B}" will typeset as "A B" not "AB". To get
% "AB" then you have to do: "\textbf{A}\textbf{B}"
% \thanks is no different in this regard, so shield the last } of each \thanks
% that ends a line with a % and do not let a space in before the next \thanks.
% Spaces after \IEEEmembership other than the last one are OK (and needed) as
% you are supposed to have spaces between the names. For what it is worth,
% this is a minor point as most people would not even notice if the said evil
% space somehow managed to creep in.



% The paper headers
%\markboth{Journal of \LaTeX\ Class Files,~Vol.~6, No.~1, January~2007}%
%{Shell \MakeLowercase{\textit{et al.}}: Bare Demo of IEEEtran.cls for Journals}
% The only time the second header will appear is for the odd numbered pages
% after the title page when using the twoside option.
%
% *** Note that you probably will NOT want to include the author's ***
% *** name in the headers of peer review papers.                   ***
% You can use \ifCLASSOPTIONpeerreview for conditional compilation here if
% you desire.




% If you want to put a publisher's ID mark on the page you can do it like
% this:
%\IEEEpubid{0000--0000/00\$00.00~\copyright~2007 IEEE}
% Remember, if you use this you must call \IEEEpubidadjcol in the second
% column for its text to clear the IEEEpubid mark.



% use for special paper notices
%\IEEEspecialpapernotice{(Invited Paper)}




% make the title area
\maketitle

\begin{abstract}
 \boldmath Multiscale models that link sub-cellular, cellular and multicellular components offer powerful insights in disease development. Such models need a realistic set of parameters to represent the physical and chemical mechanisms at the sub-cellular and cellular levels to produce high fidelity multicellular outcomes. However, determining correct values for some of the parameters is often difficult and expensive using high-throughput microfluidic approaches. This work presents an alternative approach that estimates cellular parameters from spatiotemporal data produced from bioengineered multicellular \textit{in vitro} experiments. Specifically, we apply a search technique to an integrated cellular and multicellular model of retinal pigment epithelial (RPE) cells to estimate the binding rate and auto-regulation rate of Vascular Endothelial Growth Factor (VEGF). Understanding VEGF regulation is critical in treating age-related macular degeneration and many other diseases. The method successfully identifies realistic values for autoregulatory cellular parameters that reproduce the spatiotemporal \textit{in vitro} experimental data.


\end{abstract}


% IEEEtran.cls defaults to using nonbold math in the Abstract.
% This preserves the distinction between vectors and scalars. However,
% if the journal you are submitting to favors bold math in the abstract,
% then you can use LaTeX's standard command \boldmath at the very start
% of the abstract to achieve this. Many IEEE journals frown on math
% in the abstract anyway.

% Note that keywords are not normally used for peerreview papers.
\renewcommand\IEEEkeywordsname{Key Words}
\begin{IEEEkeywords}
Systems Biology, Agent-Based Model, Parameter Estimation, Optimization, Micropatterning, Vascular endothelial growth factor (VEGF).
\end{IEEEkeywords}








% For peer review papers, you can put extra information on the cover
% page as needed:
% \ifCLASSOPTIONpeerreview
% \begin{center} \bfseries EDICS Category: 3-BBND \end{center}
% \fi
%
% For peerreview papers, this IEEEtran command inserts a page break and
% creates the second title. It will be ignored for other modes.
%\IEEEpeerreviewmaketitle



\section{INTRODUCTION}
\IEEEPARstart
Introduce the science and the relevancy of the work here.

% The very first letter is a 2 line initial drop letter followed
% by the rest of the first word in caps.
%
% form to use if the first word consists of a single letter:
% \IEEEPARstart{A}{demo} file is ....
%
% form to use if you need the single drop letter followed by
% normal text (unknown if ever used by IEEE):
% \IEEEPARstart{A}{}demo file is ....
%
% Some journals put the first two words in caps:
% \IEEEPARstart{T}{his demo} file is ....
%
% Here we have the typical use of a "T" for an initial drop letter
% and "HIS" in caps to complete the first word.


\section{Multicellular Experiment and Model}

The experiment from which the unknown parameter values are derived employs bioengineered micropatterning techniques. The micropatterns form a regular arrangement of circular 2D ‘patches’ populated with cells surrounded by an exposed substrate. The exposed regions emulate necrotic areas of the retinal tissue that result from repeated exposure to reactive oxidative species, triggering neovascularization and exudative AMD \cite{Chopdar2003Age}. Recreating these regular spatially organized cellular configurations is essential to understanding the impact of local cell-cell and cell-environment interactions on VEGF autoregulation. 

In the experimental study, described in \cite{qanitabaker:Vargis2014Effect}, the bioengineered circular micro patterns were employed to control the extent of cell-cell interactions, which occur within the patch, and cell-environment interactions, which occur at the perimeter. Several patch sizes were used in this study ( \SI{100}{\micro\metre}, \SI{200}{\micro\metre}, \SI{300}{\micro\metre}, and \SI{400}{\micro\metre}) to sample the proportion of cell-cell and cell-environment interactions in each experiment. Such sampling constrains the possible parameter values. Each patch was seeded with retinal pigment epithelial (RPE) cells and grown in a cellular culture. As the cells grew, the VEGF per cell was measured at regular intervals: 4, 24, 30 48, 54, 72 hours. To measure the VEGF per cell, enzyme-linked immunosorbent assay (ELISA) was used to determine the total VEGF contained within the cell culture, and the number of cells per patch was determined by image analysis proceeded by staining. Figure~\ref{in_vitro_experiment}(a) (taken from \cite{qanitabaker:Vargis2014Effect}) illustrates the stained patches at 72 hours. Experiments were repeated ten times and averaged. The final spatiotemporal data produced is illustrated in Figure~\ref{in_vitro_experiment}(b) and forms the target prediction for the computational model simulation.

The bioengineered experiments were simulated using a hybrid agent-based approach, which is an extension of \textsl{iDynoMiCs} framework developed by the Kreft group at University of Birmingham \cite{qanitabaker:Lardon2011IDynoMiCS}. This model was selected because of its extensibility and easy of use. All inputs to the model such as parameter values and initial condition are easily specified using an XML document called the protocol file. Hybrid models integrate discrete components to represent the cells and continuous equations to represent biochemical reactions and diffusion. Each cell is a spherical particle that grows by consuming nutrient and accumulating biomass volume; when the volume exceeds twice the initial volume, cell division is simulated by splitting the particle into two. Particles can secrete and uptake soluble biochemicals (such as VEGF) which diffuse through the domain; regulatory reactions that model interactions among intracellular and inter-cellular proteins become PDEs. The simulation interlaces cellular growth and movement (implemented by relaxing forces between particles) with biochemical redistribution (implemented by solving the PDEs). Random noise disrupts cellular movement and the division volume to represent the inherent stochasticity of the biological processes.

The setup of the simulations replicate the experimental conditions and units of the \textit{in vitro} experiments. The 2D domain size of each simulation is \SI{2400}{\micro\metre} by \SI{2400}{\micro\metre}, initial cell size is set to $80 \mu m^2$ and the doubling time due to growth is set at 36 hours. Each simulation begins with multiple RPE cells distributed randomly at the same density and with the patch pattern. The simulation replicates the first 72 hours of the \textit{in vitro} experiments. Illustrations of the simulated experiments are shown in Figure~\ref{simulation_examples}.

This framework is inherently multiscale in that the parameters that control the low-level mechanisms at the cellular level, e.g., growth, the VEGF secretion rate and autoregulation, determine the cell population and VEGF concentration over the complete multicellular domain. Figure~\ref{simulation_examples} illustrate the VEGF distributions in the domain. To compute the VEGF concentration per cell, the total VEGF is computed over the whole domain, while the number of cells is directly determined by the simulator. This approach intrinsically includes the quantitative spatiotemporal control effects as the cells grow, secret VEGF which diffuses over the domain. Moreover, the simulations provide insight into the spatial VEGF gradients within and between patches, unavailable in \textit{in vitro} studies.

%\cite{qanitabaker:Cao2007Spatiotemporal}

% \begin{figure}[!t]
%  \centering
%
% \begin{subfigure}{.5\textwidth}
%   \centering
%   \includegraphics[width=.8\linewidth]{./figures/in_vitro_crop.png}
%   \caption{(a) Patches of stained RPE cells at 72 hours for each patch size. \cite{qanitabaker:Vargis2014Effect}.}
%
% \end{subfigure}%
%
% \begin{subfigure}{.5\textwidth}
%   \centering
%   \includegraphics[width=.8\linewidth]{./figures/Results/In-Vitro.png}
%   \caption{(b) Time course of VEGF expression per cell measured at 4, 24, 30, 48, and 72 h ( data for each time from the \textit{in vitro} \cite{qanitabaker:Vargis2014Effect} ).}
% \end{subfigure}%
%\label{in_vitro_experiment}
%\end{figure}







\begin{figure}
 \begin{center}
  \begin{tabular}{cc}
   \includegraphics[width=0.40\textwidth]{./figures/in_vitro_crop.png} \\
   (a)\\
    \includegraphics[width=0.40\textwidth]{./figures/Results/In-Vitro.png} \\
   (b)   
   \end{tabular}
   \end{center}

\caption{(a) Patches of stained RPE cells at 72 hours for each patch size. \cite{qanitabaker:Vargis2014Effect}. (b) Time course of VEGF expression per cell measured at 4, 24, 30, 48, and 72 h ( data for each time from the \textit{in vitro} \cite{qanitabaker:Vargis2014Effect} ). }
  \vspace{+1mm}
\label{in_vitro_experiment}
\end{figure}


 The autoregulation of VEGF secretion is described in Equation~\ref{VEGF_autoregulation} as a function of biomass $M$ and local VEGF concentration $V$. The diffusion coefficient of VEGF, $D_{V}$, is set to $5.8 \times 10^{-11} m^{2} s^{-1}$ given in microfluidic experiments from \cite{qanitabaker:Shin2012Microfluidic}.

 \begin{equation}
 \frac{\partial V}{\partial t}=D_{V}\bigtriangledown^{2} V+ \mu _{V}  \frac{K}{ \beta V+K} M
 \label{VEGF_autoregulation}
 \end{equation}

Over time, the RPE cells grow based on a doubling time of 36 hours \cite{qanitabaker:Bryckaert2000Regulation}, which determines the growth rate parameter $\mu _{M}$. Since nutrient is unlimited and cell crowding is not an issue within the 72 hour time line, we applied first order kinetics for cell growth as shown in Equation \ref{Cell_Growth}.\\


\begin{equation}
\frac{\partial M}{\partial t}=   \mu _{M} M
\label{Cell_Growth}
\end{equation}

Table \ref{parameters} summarizes the description of the parameters used in the equations above and identifies the known parameters and those that need to be determined by the method introduced in this paper.

\begin{table}[ht]
\caption{ Known and unknown parameter descriptions} % title of Table
\centering
\begin{footnotesize}
\begin{tabular}{l l l}
\hline
Parameter   &  Value & Description\\ \hline \hline
%\\ [1ex]      % [1ex] adds vertical space
$D_{V}$     & $5.8 \times 10^{-11} m^{2} s^{-1}$ & Diffusion coefficient \\
$\mu _{M}$  &  1.0194   $ hour^{-1}$                 & Max. growth rate for RPE cells \\
[1ex]      % [1ex] adds vertical space
\hline
%\\ [1ex]      % [1ex] adds vertical space
$K$       &  \textsl{Unknown}                  & Auto regulation rate \\
$\mu _{V}$ & \textsl{Unknown}                   & Secretion rate \\
$\beta $    &  \textsl{Unknown}                  & Binding affinity \\
[1ex]      % [1ex] adds vertical space
 

 
 
\hline
\end{tabular}
\end{footnotesize}
\label{parameters}
\end{table}

\begin{figure}
 \begin{center}
  \begin{tabular}{cc}
   \includegraphics[width=0.20\textwidth]{./figures/400_one.png} &   \includegraphics[width=0.20\textwidth]{./figures/Patch400.png} \\
   (a) & (b) \\
   \includegraphics[width=0.20\textwidth]{./figures/O400VEGF solute 072.png} &  \includegraphics[width=0.20\textwidth]{./figures/C400VEGF solute 072.png} \\
   (c) & (d)
   \end{tabular}
   \end{center}

\caption{(a) Closeup of initial condition of a \SI{400}{\micro\metre} patch (b) Experiment with \SI{400}{\micro\metre}, 3 patches in each side. (c) Closeup of the VEGF distribution after 72 hours, (d) VEGF distribution over whole domain after 72 hours.  }
  \vspace{+1mm}
\label{simulation_examples}
\end{figure}




\section{METHOD}

Explain the research method here.  How did we study the problem?




\section{RESULTS}


\begin{figure}[!t]
\centering
\includegraphics[width=3in]{./figures/Results/ki.png}

\caption{Error based on different $K$ values with iteration number $It$}
\label{Ki}
\end{figure}




\begin{figure}[!t]
\centering
\includegraphics[width=3in]{./figures/Results/KiB1.png}

\caption{The heat map of error between $K$ and VEGF binding coefficient/rate ($\beta$) determined from the first round sweep.}
\label{KiB1}
\end{figure}



\begin{figure}[!t]
\centering
\includegraphics[width=3in]{./figures/Results/KiB2.png}

\caption{The heat map of error between $K$ and VEGF binding coefficient/rate ($\beta$) determined from the second round sweep.}
\label{KiB2}
\end{figure}


\begin{figure}[!t]
\centering
\includegraphics[width=3in]{./figures/Results/KiB3.png}

\caption{The heat map of error between $K$ and VEGF binding coefficient/rate ($\beta$) from the third round sweep.}
\label{KiB3}
\end{figure}


Initially the method was applied to a single parameter, $K$. Figure \ref{Ki} shows the error values for five sweep iterations over $K$. In each iteration, a parameter sweep for each patch size was performed. In this run, the values of VEGF secretion rate $\mu _{V}$ and VEGF binding rate $\beta$ were set to 0.09 pg/ml and 1.0, respectively. In iteration 1 ($It1$) the parameters were swept from 0.1 to 0.9, then the best value was chosen to determine the sweeping range for iteration 2 ($It2$). Over repeated sweeps the range of possible valid values for $K$ was greatly reduced. After five iterations of sweep processes, the best $K$ value obtained was 0.13 and the associated error was 1.01. This search considered many potential solutions and all but one were rejected as sub-optimal.

Figure \ref{KiB1} shows the error heat map of the first sweep iteration ($It1$) over two parameters ($K$ and $\beta$). For this run, the VEGF secretion rate $\mu _{V}$ was set to 0.078 pg/ml/hour. As shown, the error is lowest when the VEGF binding rate ($\beta$) is less than 0.3 and the $K$ value is greater than 0.03. Figure \ref{KiB2} shows the second sweep ($It2$), which has adjusted sweeping ranges for $K$ and $\beta$. Successive iterations refine the parameter values, as shown through reduced error in Figure \ref{KiB3}, which shows the third sweep ($It3$). The same sweeping processes was also performed between ($\mu _{V}$ and $K$) and ($\mu _{V}$ and $\beta$) (data not shown).

To show how the concentration of VEGF changes over time in the simulation, VEGF expression levels were calculated at various time intervals (4, 24, 30, 48, 54, 72 h). Figure \ref{In-Vitro_Data} shows the data from \textit{in vitro} and Figure \ref{AfterOpt} the \textit{in silico} model, $K$, $\mu _{V}$, and $\beta$ are set to 0.2, 0.07874 and 0.899, respectively, which were determined from a near-optimal solution discovered by the search method. The error associated with these results is 0.925. %as shown in the results obtained from the \textit{in vitro} experiments and the \textit{in silico} model outputs in Figure \ref{In-Vitro_Data}, and Figure \ref{AfterOpt}, respectively,
As shown in figure \ref{In-Vitro_Data} and \ref{AfterOpt}, in both \textit{in silico} and \textit{in vitro}, RPE cells in the smaller patches expressed higher levels of VEGF per cell. This indicates that these cells function to maintain a consistent level of VEGF within their local microenvironment. Cells in smaller patches respond by expressing higher amounts of VEGF because the VEGF expression levels are dominated by cell-environment interactions. In contrast, larger patches maintain lower basal levels of VEGF because cell-cell auto-inhibitory regulation dominates.

Now the parameters $K$, $\mu_V$ and $\beta$ have been determined, we replace the variables with their values in Equation~\ref{VEGF_autoregulation} and set the mass $M$ to that of a single cell ($M = 25.95 pg$) resulting in the VEGF autoregulatory function for RPE cells being:

\begin{equation}
 \frac{d V}{d t}= 0.07874 \times \frac{0.2}{ 0.899 V+0.2} \times 25.95
 \label{VEGF_final}
 \end{equation}

This function is plotted in Figure~\ref{finalGraph}.

\begin{figure}[!t]
\includegraphics[width=.8\linewidth]{./figures/Results/Final_Result.png}
   \caption{The VEGF autoregulatory function of RPE cells showing how the secretion rate of VEGF is down regulated as a function of the VEGF in the microenvironment.}
   \label{finalGraph}
   \end{figure}

  \begin{figure}[!t]
  \centering


 \begin{subfigure}{.5\textwidth}
   \centering
   \includegraphics[width=.8\linewidth]{./figures/Results/In-Vitro.png}
   \caption{Time course of VEGF expression per cell measured at 4, 24, 30, 48, and 72 h (data from the \textit{in vitro} work \cite{qanitabaker:Vargis2014Effect} ).}
   \label{In-Vitro_Data}
 \end{subfigure}%


  \begin{subfigure}{.5\textwidth}
    \centering
    \includegraphics[width=.8\linewidth]{./figures/Results/After.png}
    \caption{Time course of VEGF expression per cell measured at 4, 24, 30, 48, and 72 h (data from the \textit{in silico} model after optimization).}
    \label{AfterOpt}
  \end{subfigure}%

  \end{figure}






%\section{SUMMARY}

\section{CONCLUSION}



One of the main challenges in the computational modeling of biological systems is the determination of the models' parameters. This problem is particularly acute with multiscale models that are gaining in popularity due to their realism. In this work, we identified the parameter values of a cellular regulatory mechanism using spatiotemporal multicellular data. While the problem of finding parameter values that describe the VEGF autoregulatory mechanism of RPE cells is simple, this problem domain serves and a proof-of-concept for the overall method. Most importantly, the method demonstrates that it is possible to utilize data at one scale to determine parameter values at a different scale.

In the work presented here, thousands of simulations were performed as the search method explored the parameter space. For each potential solution, multiple simulations were needed over each experimental case (different patch sizes) and because of the need for repeats due to model stochasticity. For the simple 2D RPE model, each simulation took less than a minute and so the process could be completed quickly. In general, the method can rapidly become infeasible as the number of unknown parameters grows, domains become larger and more complex, and the number of specific experimental cases grows.

A recently developed hybrid simulation system called Biocellion \cite{Kang2014Biocellion} could be used instead of the iDynamics to speedup the rate of simulations. Biocellion can utilize thousands of processing nodes and rapidly simulate complex models of billions of cells. The search method utilized in this paper was chosen for its simplicity and insights on the local error surfaces provided by the sweeping process. However, the fitting method is independent of the method employed to search the parameter space. Alternative methods of combinatorial optimization may improve performance.

One problem all fitting methods must deal with is under or over fitting the model. There is a possibility that the fitting problem may be under-constrained for lack of data. This problem will be explored in the RPE domain by expanding the data set to include additional studies with VEGF agonists and alternative pattern arrangements.
 
To extend the method to other domains, alternative error functions can be employed that measure discrepancies over a diversity of spatiotemporal features which quantify both the experimental observations and simulator outcomes. For instance, in \cite{Ruusuvuori2014Quantitative} image processing is applied to bright field time-lapse images of growing yeast colonies to extract trajectories of many visual features including volume, roughness, dominant frequency etc. The same features could be extracted from the morphologies of simulated colonies and used to fit parameters of the yeast model.




%%//Future works  In this model error minimization is used to determine model parameters that optimally fit the data are. The best model parameters should be found by choosing, among suboptimal parameters, those that match criteria other than the ones used to fit the model \cite{qanitabaker:Slezak2010When}. In the %future, other biological criteria need to be addressed in finding data and optimization approach form a new complex system and point to the need of a theory that addresses this problem more generally.
% use section* for acknowledgement
\section*{Acknowledgment}

This work was supported by the Luxembourg Centre for Systems Biomedicine, the University of Luxembourg and the Institute for Systems Biology, Seattle, USA. Research reported in this publication was partially supported by the National Institute Of General Medical Sciences of the National Institutes of Health under Award Number P50GM076547. The content is solely the responsibility of the authors and does not necessarily represent the official views of the National Institutes of Health.


\bibliographystyle{plain}

\begin{small}

\bibliography{qanitabaker-retinal,qanitabaker-fittingthree,nicholasflann}


\end{small}







\end{document}


